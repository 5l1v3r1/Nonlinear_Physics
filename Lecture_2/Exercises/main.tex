\documentclass{article}
\usepackage[utf8]{inputenc}
\usepackage{multicol}
\usepackage[]{amsmath}
\usepackage[]{amssymb}

\title{Physique non-linéaire}
\author{Jean-Christophe Loiseau}
\date{Décembre 2017}

\usepackage{natbib}
\usepackage{graphicx}

\begin{document}

\maketitle

\section*{Exercice 1: Systèmes du premier ordre}

À l'aide d'analyses de stabilité linéaire, déterminez la stabilité des différents points fixes $x^*$ des systèmes suivants.

\begin{multicols}{2}
    \begin{itemize}
        \item $\dot{x} = x(1-x)$
        \item $\dot{x} = \ln(x)$
        \item $\dot{x} = ax - 4x^3$ où $a$ peut être positif, négatif ou nul. Discutez des trois cas de figures.
        \item $\dot{x} = -ax \ln(bx)$ avec $a$ et $b$ positifs.
    \end{itemize}
\end{multicols}

\noindent Si une analyse de stabilité s'avère non-applicable car $f^{\prime}(x^*)=0$, déterminez la stabilité du point fixe à l'aide d'arguments géométriques.

\section*{Exercice 2: Systèmes du deuxième ordre}

Considérons les deux systèmes suivants:

\begin{multicols}{2}
    \begin{itemize}
        \item $\dot{x} = x - x^3$ \\ $\dot{y} = -y$
        \item $\dot{x} = x^2 -y$ \\ $\dot{y} = x -y$
    \end{itemize}
\end{multicols}

\noindent Pour chacun d'eux, trouvez les points fixes, déterminez leur stabilité et classez les. Ensuite, tracez qualitativement sur un graphique les isoclines ($\dot{x} = 0$ et $\dot{y} = 0$) ainsi qu'un plausible portrait de phase. Expliquez en quelques mots la dynamique de ces deux systèmes.

\end{document}
